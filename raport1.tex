\documentclass[a4paper,onecolumn,twoside,10pt]{article}%{mwrep}

\usepackage{times}
\usepackage[utf8x]{inputenc}
\usepackage[T1]{fontenc}
\usepackage[polish]{babel}
\usepackage{lmodern} %Type1-font for non-english texts and characters
\usepackage{setspace}
\usepackage{enumitem}
\usepackage{subcaption}

\usepackage{booktabs} % Dla ładniejszych linii poziomych (\toprule, \midrule)
\usepackage{colortbl} % Dla kolorowania wierszy
\usepackage{array}    % Użyteczne z p{...}


\usepackage[cmex10]{amsmath}

%% Packages for Graphics & Figures %%%%%%%%%%%%%%%%%%%%%%%%%%

%\usepackage{bmpsize}

\usepackage{graphicx} %%For loading graphic files
\usepackage[pdf]{pstricks}
\usepackage{pst-all}
\usepackage{moredefs}
%\usepackage{auto-pst-pdf}
%\usepackage{auto-pst-pdf}
\usepackage[crop=off]{auto-pst-pdf}

\usepackage{fancyhdr}
\usepackage{url}
\usepackage{float}
\usepackage{color}
\usepackage{xcolor}

\usepackage{multirow}

%\usepackage{epstopdf}
\definecolor{light-gray}{gray}{0.4}
\definecolor{mauve}{rgb}{0.88, 0.69, 1.0}
\definecolor{pakistangreen}{rgb}{0.0, 0.4, 0.0}
\definecolor{pearl}{rgb}{0.94, 0.92, 0.84}
\definecolor{whitesmoke}{rgb}{0.96, 0.96, 0.96}
\definecolor{gray-pp}{rgb}{0.13, 0.6, 0.82}

\usepackage{colortbl}
\usepackage{listings}
\lstset{
	basicstyle=\footnotesize\ttfamily,
	columns=fullflexible,
	frame=single,
	breaklines=true,
	numbers=left, stepnumber=2, numbersep=5pt,
	numberstyle=\tiny\color{gray},
	keywordstyle=\color{blue},
	commentstyle=\color{gray-pp},
	stringstyle=\color{mauve},
	backgroundcolor = \color{whitesmoke},
	breakatwhitespace=true,
	showspaces=false,                % show spaces everywhere adding particular underscores; it overrides 'showstringspaces'
	showstringspaces=false,          % underline spaces within strings only
	showtabs=false,                  % show tabs within strings adding particular underscores
	postbreak=\mbox{\textcolor{red}{$\hookrightarrow$}\space}
}



\cfoot{-~\thepage \textcolor{light-gray}{~| Strona~-}}

\hyphenpenalty=10000		% nie dziel wyrazów zbyt często
\clubpenalty=10000			% kara za sierotki
\widowpenalty=10000			% nie pozostawiaj wdów
\brokenpenalty=10000		% nie dziel wyrazów między stronami
\exhyphenpenalty=999999		% nie dziel słów z myślnikiem
\righthyphenmin=3			% dziel minimum 3 litery

\tolerance=4500
\pretolerance=250
\hfuzz=1.5pt
\hbadness=1450

\sloppy						% umacnia pozycję prawego marginesu

\setlength{\textwidth}{\paperwidth}
\addtolength{\textwidth}{-5cm}
\setlength{\textheight}{\paperheight}
\addtolength{\textheight}{-5cm}
\setlength{\oddsidemargin}{0cm}
\setlength{\evensidemargin}{0cm}
\topmargin -1.25cm
\footskip 1.4cm

\linespread{1.3}

\begin{document}
	\raggedbottom 
	%\input{hypernation}
%\input{title-page}
\setlength\extrarowheight{2pt}
\begin{table}[ht]
	\centering
	%\resizebox{\textwidth}{!}{%
		\begin{tabular}{|p{5cm}|p{7cm}|p{2cm}|}
			\hline
			%\rowcolor{gray}
			%\multicolumn{3}{|c|}{\textcolor[rgb]{1,1,1}{Scalenie trzech kolumn}}\\
			
			\multicolumn{2}{|c|}{\cellcolor{gray-pp}\textcolor[rgb]{1,1,1}{Politechnika Poznańska}}  & \multicolumn{1}{c|}{\multirow{3}{*}{\resizebox{15mm}{!}{\includegraphics{PP_znak_konturowy_CMYK.pdf}}}}\\ 
			\multicolumn{2}{|c|}{\cellcolor{gray-pp}\textcolor[rgb]{1,1,1}{Wydział Automatyki, Robotyki i Elektrotechniki}} & \\ 
			\multicolumn{2}{|c|}{\cellcolor{gray-pp}\textcolor[rgb]{1,1,1}{Instytut Robotyki i Inteligencji Maszynowej}} & \\ 
			\hline 
			\multicolumn{1}{|c|}{Dz>AiR>Sem5} & \multicolumn{1}{c|}{Napędy przekształtnikowe (NP)} & \multicolumn{1}{c|}{2025/26 (s.zim.)} \\
			\hline
			\textbf{Skład osobowy:} \par Zuzanna Andrzejak 159522 \par Jan Andrzejewski 159512 \par Mateusz Banaszak 159416 \par Piotr Bednarek 159701 & 
			\textbf{Silnik elektryczny i jego model obwodowy. Siły: Lorentz'a, elektrodynamiczna, elektromotoryczna. Zbieżność modelu.} 
			& Data wyk.:\par 20.11.2025\\
			\hline
			Grupa 1  & Ćwiczenie 1 & Zajęcia 1 \\
			\hline
		\end{tabular}%
		%}
\end{table}	
\setlength\extrarowheight{0pt}
\vspace{1.5cm}
\tableofcontents
\newpage

\section{Wprowadzenie}

Ćwiczenie laboratoryjne miało na celu analizę modelu obwodowego silnika obcowzbudnego komutatorowego prądu stałego \cite{ekurs}. Model zastosowany podczas ćwiczenia był modelem zidealizowanym i nie stanowił dokładnego odwzorowania rzeczywistego ukłądu napędowego. 

Ćwiczenie przeprowadzono w sali C3 w budynku A22b, wyposażonej w zestawy laboratoryjne napędowe, panele sterowania oraz przyrządy pomiarowe. W trakcie zajęć wykorzystano równień środowisko symulacyjne MATLAB wraz z Simulink. 
\vspace{1.5cm}

\section{Wstęp teoretyczny}

Model zrealizowany podczas ćwiczenia można w przybliżeniu opisać z wykorzystaniem podstawowych zjawisk fizycznych. Obejmują one przede wszystkim zjawiska elektromagnetyczne, elektromotoryczne oraz elektrodynamiczne, które określają zależności między przepływem prądu, strumieniem magnetycznym a momentem elektromagnetycznym oraz momentami wynikającymi z dynamiki ruchomych części silnika.

\subsection{Siła Lorentz'a}
Za generowanie siły w silniku prądu stałego odpowiada siła Lorentz'a, która określa oddziaływanie pola elektrycznego i magnetycznego na poruszający się ładunek. Jest określana równaniem:

\begin{equation}
	\vec{F}_{l} = q(\vec{E} + \vec{v} \times \vec{B})
	\label{eq:lorentz}
\end{equation}
gdzie:
$\vec{F}$ to wektor siły Lorentz'a (N),
$q$ to ładunek elektryczny cząstki (C),
$\vec{E}$ to wektor natężenia pola elektrycznego (V/m),
$\vec{v}$ to wektor prędkości cząstki (m/s),
$\vec{B}$ to pseudowektor indukcji magnetycznej (T).


\subsection{Siła elektrodynamiczna}

W rozważaniach dotyczących pracy silnika prądu stałego można pominąć wpływ wektora natężenia pola elektrycznego, gdyż jest ono praktycznie jednorodne. Przewodnik z prądem, umieszczony w polu magnetycznym, doświadcza siły, której kierunek i wartość zależą od ustawienia przewodnika względem pola. 
Maksymalna wartość siły występuje, gdy przewodnik jest ustawiony prostopadle do pola magnetycznego, co zapewnia konstrukcja \textbf{komutatora}. 

W tych warunkach siła Lorentza przyjmuje postać:
\begin{equation}
	F_{l} = qvB
	\label{eq:lorentz_simple}
\end{equation}

Dla przewodnika o określonej długości, w którym przepływają ładunki tworzące prąd, siła Lorentza przyjmuje postać \textbf{siły elektrodynamicznej}:
\begin{equation}
	\vec{F}_{ed} = I \, \vec{l} \times \vec{B}
	\label{eq:sila_ed}
\end{equation}
gdzie: $I$ to natężenie prądu, $\vec{l}$ to wektor długości przewodnika.


\subsection{Moment elektromagnetyczny}

Moment elektromagnetyczny w silniku prądu stałego jest generowany w wyniku działania sił elektrodynamicznych \eqref{eq:sila_ed} na uzwojenia twornika i wyraża się wzorem:
\begin{equation}
	T_e = \vec{r} \times \vec{F}_{ed} = I \left[ \vec{r} \times \left( \vec{l} \times \vec{B} \right) \right]
	\label{eq:moment_em_dlugie}
\end{equation}
gdzie: $I$ to natężenie prądu (A), $\vec{l}$ to długość uzwojenia (m), $\vec{B}$ to wektor indukcji magnetycznej (T), $\vec{r}$ to ramię siły (m).
\\

Dla uproszczonego modelu silnika prądu stałego można wyróżnić tzw. stałą momentową $k_{\Phi}$, która zależy od parametrów konstrukcyjnych silnika. Wtedy moment elektromagnetyczny wyraża się zależnością:
\begin{equation}
	T_e = i_a \, k_{\Phi}
	\label{eq:moment_Te}
\end{equation}
gdzie: $T_e$ to moment elektromagnetyczny (Nm), $i_a$ to prąd twornika (A), $k_{\Phi}$ to stała momentowa (Nm/A).


\subsection{Siła elektromotoryczna}

Siła elektromotoryczna (SEM) powstaje wskutek ruchu przewodnika w polu magnetycznym. Przewodniki wirnika przecinają linie pola, co indukuje napięcie i powoduje powstanie siły przeciwdziałającej ruchowi elektronów, równoważącej działanie sił Lorentza \eqref{eq:lorentz}. Dzięki temu moment elektromagnetyczny nie prowadzi do nieskończonego przyspieszenia wirnika.  

Wartość SEM zależy liniowo od prędkości obrotowej wirnika i parametrów konstrukcyjnych:
\begin{equation}
	\varepsilon = k_{\Phi} \, \omega_{r},
	\label{eq:sem}
\end{equation}
gdzie:
$\varepsilon$ to siła elektromotoryczna (V), $\omega_{r}$ to prędkość obrotowa wirnika (rad/s), $k_{\Phi}$ to stała napięciowa silnika (V $\cdot$ s/rad).


\subsection{Stała momentowa i napięciowa}

W teoretycznych rozważaniach współczynnik $k_{\Phi}$ pojawia się zarówno w równaniu \eqref{eq:moment_Te}, jak i w \eqref{eq:sem}. W rzeczywistości są to dwie różne stałe: w równaniu \eqref{eq:moment_Te} występuje \textbf{stała momentowa} $k_m$ (Nm/A), natomiast w \eqref{eq:sem} występuje \textbf{stała napięciowa} $k_v$ (V$\cdot$s/rad).  

Stała momentowa $k_m$ wpływa bezpośrednio na wartość momentu elektromagnetycznego $T_e$, który zależy od kąta położenia wirnika względem pola magnetycznego. Dlatego jej działanie nie jest symetryczne w całym cyklu obrotu. Natomiast stała napięciowa $k_v$, powiązana z siłą elektromotoryczną $\varepsilon$, działa symetrycznie, gdyż SEM zależy głównie od prędkości obrotowej wirnika i parametrów konstrukcyjnych, a nie od kąta położenia przewodnika w polu magnetycznym.

W praktyce, w notach katalogowych są one rozróżniane, jednak dla uproszczenia modelu symulacyjnego podczas ćwiczenia operowano jednym współczynnikiem $k_{\Phi} $, gdyż różnice między nimi są niewielkie.
\\ 

Współczynnik $k_{\Phi}$ ma kluczowe znaczenie w analizie charakterystyk dynamicznych silnika, ponieważ bezpośrednio wpływa na prędkość obrotową wirnika przy zadanym napięciu zasilania i prądzie twornika. \textbf{Równanie sterowania} w stanie ustalonym można wyprowadzić na podstawie bilansu napięć w obwodzie twornika oraz zależności elektromotorycznej siły (SEM) generowanej w silniku. Przyjmuje ono postać:

\begin{equation}
	\omega_r = \frac{U_a - R_a i_a}{k_{\Phi}}
	\label{eq:predkosc}
\end{equation}

Z równania \eqref{eq:predkosc} wynika, że przy stałym napięciu zasilania $U_a$ i prądzie twornika $i_a$, zmniejszenie wartości współczynnika $k_{\Phi}$ prowadzi do proporcjonalnego wzrostu prędkości obrotowej wirnika. Zjawisko to wynika z bezpośredniego związku między strumieniem magnetycznym a momentem elektromagnetycznym wytwarzanym przez silnik.

\vspace{1.5cm}
\section{Realizacja modelu w środowisku symulacyjnym}
% zadanie 1

Przed rzeczywistą realizacją projektu w środowisku Simulink zrealizowano model silnika prądu stałego (Rysunek \ref{fig:model}), zgodnie z metodyką przedstawioną w skrypcie do ćwiczenia \cite{ekurs}. Parametry silnika zadeklarowano w niezależnym skrypcie w środowisku MATLAB, jak przedstawiono w Listing \ref{lst:silnik}.

W modelu silnika prądu stałego (rysunek \ref{fig:model}) wprowadzono kilka uproszczeń, aby ułatwić analizę działania układu. Stałą napięciową oraz stałą momentową zastąpiono jednym współczynnikiem $k_{\Phi}$, zgodnie z omówieniem w punkcie~2.5. Zrezygnowano również z modelowania toru wymuszenia momentem obciążenia $T_l$, przyjmując, że w analizowanym przypadku zakłócenia wprowadzane przez ten człon są równe zero.



\begin{lstlisting}[language=Matlab,
	caption={Przyjęte parametry},
	label={lst:silnik}
	]
	Ra = 3.0;           				% (Ohm)
	La = 10e-3;    		% (H)
	kfi = 1.4;        			  % (Nm/A)
	J = 25e-3;    			% (kg*m^2)
	b1 = 0.06;      		 	  % (Nm*s/rad)
	tau = La/Ra;     	% (s)
	Ua = 240;
\end{lstlisting}

\begin{figure}[h!]
	\centering
	\includegraphics[width=\textwidth,trim=0cm 6cm 0cm 6cm,clip]{model.pdf}
	\caption{Model silnika w środowisku Simulink}
	\label{fig:model}
\end{figure}

Parametry podane w skrypcie z Listing \ref{lst:silnik} zostały zapisane do Workspace MATLAB, co umożliwiło ich wykorzystanie w modelu zbudowanym w środowisku Simulink. Następnie uruchomiono symulację, a dane z oscyloskopu zapisano w Workspace w celu uzyskania czytelnych przebiegów prędkości obrotowej oraz prądu twornika. 


\vspace{1.5cm}
\section{Analiza rozruchu modelu do prędkości biegu jałowego}
% zadanie 2

\begin{figure}[h!]
	\centering
	\includegraphics[width=\textwidth]{przebiegi2.pdf}
	\caption{Przebiegi czasowe prędkości obrotowej $\omega_r$ oraz prądu twornika $i_a$ w odpowiedzi na wymuszenie skokowe}
	\label{fig:przebiegi1}
\end{figure}

Na rysunku \ref{fig:przebiegi1} przedstawiona jest dynamika rozruchu modelu maszyny do prędkości biegu jałowego przy skokowym wymuszeniu. Wymuszeniem jest napięcie twornika o wartości $U_a = 240$ V. 
\\

Wartość prędkości obrotowej $\omega_r$ przy wymuszeniu skokowym szybko narasta, bez przeregulowań, do wartości $116$ rad/s. Wartość oczekiwana prędkości w stanie ustalonym może być wyznaczona z zależności opisanej wzorem \eqref{eq:predkosc}: 

\begin{equation}
	\omega_r = \frac{U_a}{k_{\Phi}} = \frac{240 \ \text{V}}{2 \ \text{V}\cdot \text{rad/s}} = 120 \ \text{rad/s}
\end{equation}

Prędkość osiągnięta w stanie ustalonym jest nieco niższa od wartości oczekiwanej, ponieważ w powyższym oszacowaniu pominięto spadek napięcia na rezystancji twornika $i_a R_a$. 
\\

Prąd twornika w chwili podania wymuszenia skokowego gwałtownie rośnie, a następnie opada, osiągając w stanie ustalonym wartość $2.9$ A. Możliwe jest oszacowanie oczekiwanej wartości maksymalnego prądu rozruchowego z zależności:

\begin{equation}
	i_{a \ \text{max}} = \frac{U_a}{R_a} = \frac{240 \ \text{V}}{3 \ \Omega} = 80 \ \text{A}
\end{equation}

Wartość ta jest jednak przeszacowana, ponieważ nie uwzględnia dynamiki obwodu twornika ani wpływu siły elektromotorycznej, która pojawia się wraz ze wzrostem prędkości obrotowej i ogranicza wartość płynącego prądu. W modelu symulacyjnym prąd twornika osiąga maksymalnie $62$ A.

W chwili przyłączenia obwodu twornika do modelu, prąd jaki popłynie przez obwód twornika $i_a$ ograniczony jest tylko rezystancją obwodu wornika $R_a$, gdyż przy $\omega_r = 0$ siła elektromotoryczna ma także wartość zerową $\varepsilon = 0$. Wraz ze wzrostem prędkości $\omega_r$, wzrastająca siła elektromotoryczna $\varepsilon$ powoduje zmniejszanie się wartości prądu twornika $i_a$ \cite{sidorowicz}. 

\vspace{1.5cm}
\section{Analiza porównawcza stanu biegu jałowego}
% zadanie 3
W celu weryfikacji uzyskanego modelu silnika prądu stałego przeprowadzono porównanie przebiegów prądu twornika, napięcia twornika oraz prędkości obrotowej z wartościami eksperymentalnymi.
 
W tym celu uruchomiono silnik za pośrednictwem przekształtnika DML, stosując odpowiednie, zgodne z warunkami symulacji, parametry procesowe oraz zachowując obowiązujące zasady BHP.
\\

Dane pomiarowe zostały zarejestrowane za pomocą oscyloskopu, a następnie zapisane w plikach \texttt{.CSV}. Na ich podstawie, przy użyciu skryptu w środowisku MATLAB, wygenerowano wykresy przedstawiające przebiegi wszystkich analizowanych wielkości.

Uzyskane charakterystyki umożliwiły przeprowadzenie analizy porównawczej w warunkach biegu jałowego oraz dostrojenie parametrów modelu symulacyjnego, takich jak współczynnik momentowy $k_{\Phi}$ oraz współczynnik tarcia wewnętrznego $b_1$. Parametry te dobrano w taki sposób, aby uzyskać zgodność przebiegów symulacyjnych z eksperymentalnymi na poziomie $\pm 5\%$.

\subsection{$\mathbf{U_a = 50}$ V}

Współczynniki $k_{\Phi}$ oraz $b_1$ zostały pozostawione bez zmian, zgodnie z parametrami symulacji zawartymi w Listing \ref{lst:silnik}. Dla wymuszenia napięciem twornika $U_a = 50$ V uzyskano zbieżność symulacji z danymi empirycznymi o wartości: 

\begin{equation}
	\frac{U_e - U_s}{U_e} \cdot 100 \% = \frac{52.5 \text{V} - 50 \text{V}}{52.5 \text{V}} \cdot 100 \% \approx 4.7\%
	\label{eq:approx_50}
\end{equation}

Na rysunku \ref{fig:sym50} został przedstawiony wykres symualcji dla danego wymuszenia, a na rysunku \ref{fig:50} wykresy oparte na wartościach uzyskanych eksperymentalnie. 

\begin{figure}[h!]
	\centering
	\includegraphics[width=0.5\textwidth]{model50.pdf}
	\caption{Przebiegi czasowe prędkości obrotowej oraz prądu twornika uzyskanie symulacyjnie}
	\label{fig:sym50}
\end{figure}


\begin{figure}[H]
	\centering
	
	\begin{subfigure}[b]{0.48\textwidth}
		\includegraphics[width=\textwidth,trim=0cm 9cm 0cm 9cm,clip]{50_current.pdf}
		\caption{Przebieg odfiltrowanego natężenia prądu twornika}
	\end{subfigure}
	\hfill
	\begin{subfigure}[b]{0.48\textwidth}
		\includegraphics[width=\textwidth,trim=0cm 9cm 0cm 9cm,clip]{50_speed.pdf}
		\caption{Przebieg odfiltrowanej prędkości obrotowej}
	\end{subfigure}
	
	\vspace{0.5cm}
	
	\begin{subfigure}[b]{0.48\textwidth}
		\includegraphics[width=\linewidth]{wykres4_50.pdf}
		\caption{Przebieg odfiltorwanego napiecia twornika}
	\end{subfigure}
	\hfill
	\begin{subfigure}[b]{0.48\textwidth}
		\includegraphics[width=\linewidth]{wykres2_50.pdf}
		\caption{Przebiegi czasowe prędkości obrotowej oraz prądu twornika}
	\end{subfigure}
	
	\vspace{0.5cm}
	
	\begin{subfigure}[b]{0.48\textwidth}
		\includegraphics[width=\linewidth]{wykres3_50.pdf}
		\caption{Przebiegi odczytane z oscyloskopu}
	\end{subfigure}
	\hfill
	\begin{subfigure}[b]{0.48\textwidth}
		\includegraphics[width=\linewidth]{wykres4_ 2okresy_50.pdf}
		\caption{Przybliżony przebieg napiecia twornika, na którym widoczne są momenty przełączania termistorów}
	\end{subfigure}
	
	\caption{Zestawienie wyników eksperymentalnych dla napięcia twornika $U_a = 50$ V}
	\label{fig:50}
\end{figure}

\subsection{$\mathbf{U_a = 100}$ V}

Współczynniki $k_{\Phi}$ oraz $b_1$ zostały pozostawione bez zmian, zgodnie z parametrami symulacji zawartymi w Listing \ref{lst:silnik}. Dla wymuszenia napięciem twornika $U_a = 100$ V uzyskano zbieżność symulacji z danymi empirycznymi o wartości: 
\begin{equation}
	\frac{U_e - U_s}{U_e} \cdot 100 \% = \frac{102 \text{V} - 100 \text{V}}{102 \text{V}} \cdot 100 \% \approx 1.9 \%
	\label{eq:approx_100}
\end{equation}

Na rysunku \ref{fig:sym100} został przedstawiony wykres symualcji dla danego wymuszenia, a na rysunku \ref{fig:100} wykresy oparte na wartościach uzyskanych eksperymentalnie. 

\begin{figure}[h!]
	\centering
	\includegraphics[width=0.5\textwidth]{model100.pdf}
	\caption{Przebiegi czasowe prędkości obrotowej oraz prądu twornika uzyskanie symulacyjnie}
	\label{fig:sym100}
\end{figure}


\begin{figure}[H]
	\centering
	
	\begin{subfigure}[b]{0.48\textwidth}
		\includegraphics[width=\textwidth,trim=0cm 9cm 0cm 9cm,clip]{100_current.pdf}
		\caption{Przebieg odfiltrowanego natężenia prądu twornika}
	\end{subfigure}
	\hfill
	\begin{subfigure}[b]{0.48\textwidth}
		\includegraphics[width=\textwidth,trim=0cm 9cm 0cm 9cm,clip]{100_speed.pdf}
		\caption{Przebieg odfiltrowanej prędkości obrotowej}
	\end{subfigure}
	
	\vspace{0.5cm}
	
	\begin{subfigure}[b]{0.48\textwidth}
		\includegraphics[width=\linewidth]{wykres4_100.pdf}
		\caption{Przebieg odfiltorwanego napiecia twornika}
	\end{subfigure}
	\hfill
	\begin{subfigure}[b]{0.48\textwidth}
		\includegraphics[width=\linewidth]{wykres2_100.pdf}
		\caption{Przebiegi czasowe prędkości obrotowej oraz prądu twornika}
	\end{subfigure}
	
	\vspace{0.5cm}
	
	\begin{subfigure}[b]{0.48\textwidth}
		\includegraphics[width=\linewidth]{wykres3_100.pdf}
		\caption{Przebiegi odczytane z oscyloskopu}
	\end{subfigure}
	\hfill
	\begin{subfigure}[b]{0.48\textwidth}
		\includegraphics[width=\linewidth]{wykres4_2okresy_100.pdf}
		\caption{Przybliżony przebieg napiecia twornika, na którym widoczne są momenty przełączania termistorów}
	\end{subfigure}
	
	\caption{Zestawienie wyników eksperymentalnych dla napięcia twornika $U_a = 100$ V}
	\label{fig:100}
\end{figure}

\subsection{$\mathbf{U_a = 201}$ V}

Współczynniki $k_{\Phi}$ oraz $b_1$ zostały pozostawione bez zmian, zgodnie z parametrami symulacji zawartymi w Listing \ref{lst:silnik}. Dla wymuszenia napięciem twornika $U_a = 201$ V uzyskano zbieżność symulacji z danymi empirycznymi o wartości: 
\begin{equation}
	\frac{U_e - U_s}{U_e} \cdot 100 \% = \frac{206 \text{V} - 201 \text{V}}{206 \text{V}} \cdot 100 \% \approx 2.4 \%
	\label{eq:approx_201}
\end{equation}

Na rysunku \ref{fig:sym201} został przedstawiony wykres symualcji dla danego wymuszenia, a na rysunku \ref{fig:201} wykresy oparte na wartościach uzyskanych eksperymentalnie. 

\begin{figure}[h!]
	\centering
	\includegraphics[width=0.5\textwidth]{model201.pdf}
	\caption{Przebiegi czasowe prędkości obrotowej oraz prądu twornika uzyskanie symulacyjnie}
	\label{fig:sym201}
\end{figure}


\begin{figure}[H]
	\centering
	
	\begin{subfigure}[b]{0.48\textwidth}
		\includegraphics[width=\textwidth,trim=0cm 9cm 0cm 9cm,clip]{201_current.pdf}
		\caption{Przebieg odfiltrowanego natężenia prądu twornika}
	\end{subfigure}
	\hfill
	\begin{subfigure}[b]{0.48\textwidth}
		\includegraphics[width=\textwidth,trim=0cm 9cm 0cm 9cm,clip]{201_speed.pdf}
		\caption{Przebieg odfiltrowanej prędkości obrotowej}
	\end{subfigure}
	
	\vspace{0.5cm}
	
	\begin{subfigure}[b]{0.48\textwidth}
		\includegraphics[width=\linewidth]{wykres4_201.pdf}
		\caption{Przebieg odfiltorwanego napiecia twornika}
	\end{subfigure}
	\hfill
	\begin{subfigure}[b]{0.48\textwidth}
		\includegraphics[width=\linewidth]{wykres2_201.pdf}
		\caption{Przebiegi czasowe prędkości obrotowej oraz prądu twornika}
	\end{subfigure}
	
	\vspace{0.5cm}
	
	\begin{subfigure}[b]{0.48\textwidth}
		\includegraphics[width=\linewidth]{wykres3_201.pdf}
		\caption{Przebiegi odczytane z oscyloskopu}
	\end{subfigure}
	\hfill
	\begin{subfigure}[b]{0.48\textwidth}
		\includegraphics[width=\linewidth]{wykres4_2okresy_201.pdf}
		\caption{Przybliżony przebieg napiecia twornika, na którym widoczne są momenty przełączania termistorów}
	\end{subfigure}
	
	\caption{Zestawienie wyników eksperymentalnych dla napięcia twornika $U_a = 201$ V}
	\label{fig:201}
\end{figure}

\subsection{$\mathbf{U_a = 240}$ V}

Współczynniki $k_{\Phi}$ oraz $b_1$ zostały pozostawione bez zmian, zgodnie z parametrami symulacji zawartymi w Listing \ref{lst:silnik}. Dla wymuszenia napięciem twornika $U_a = 240$ V uzyskano zbieżność symulacji z danymi empirycznymi o wartości: 

\begin{equation}
	\frac{U_e - U_s}{U_e} \cdot 100 \% = \frac{248 \text{V} - 240 \text{V}}{248 \text{V}} \cdot 100 \% \approx 3.2 \%
	\label{eq:approx_240}
\end{equation}

Na rysunku \ref{fig:sym240} został przedstawiony wykres symualcji dla danego wymuszenia, a na rysunku \ref{fig:240} wykresy oparte na wartościach uzyskanych eksperymentalnie. 

\begin{figure}[h!]
	\centering
	\includegraphics[width=0.5\textwidth]{model240.pdf}
	\caption{Przebiegi czasowe prędkości obrotowej oraz prądu twornika uzyskanie symulacyjnie}
	\label{fig:sym240}
\end{figure}


\begin{figure}[H]
	\centering
	
	\begin{subfigure}[b]{0.48\textwidth}
		\includegraphics[width=\textwidth,trim=0cm 9cm 0cm 9cm,clip]{240_current.pdf}
		\caption{Przebieg odfiltrowanego natężenia prądu twornika}
	\end{subfigure}
	\hfill
	\begin{subfigure}[b]{0.48\textwidth}
		\includegraphics[width=\textwidth,trim=0cm 9cm 0cm 9cm,clip]{240_speed.pdf}
		\caption{Przebieg odfiltrowanej prędkości obrotowej}
	\end{subfigure}
	
	\vspace{0.5cm}
	
	\begin{subfigure}[b]{0.48\textwidth}
		\includegraphics[width=\linewidth]{wykres4_240.pdf}
		\caption{Przebieg odfiltorwanego napiecia twornika}
	\end{subfigure}
	\hfill
	\begin{subfigure}[b]{0.48\textwidth}
		\includegraphics[width=\linewidth]{wykres2_240.pdf}
		\caption{Przebiegi czasowe prędkości obrotowej oraz prądu twornika}
	\end{subfigure}
	
	\vspace{0.5cm}
	
	\begin{subfigure}[b]{0.48\textwidth}
		\includegraphics[width=\linewidth]{wykres3_240.pdf}
		\caption{Przebiegi odczytane z oscyloskopu}
	\end{subfigure}
	\hfill
	\begin{subfigure}[b]{0.48\textwidth}
		\includegraphics[width=\linewidth]{wykres4_2okresy_240.pdf}
		\caption{Przybliżony przebieg napiecia twornika, na którym widoczne są momenty przełączania termistorów}
	\end{subfigure}
	
	\caption{Zestawienie wyników eksperymentalnych dla napięcia twornika $U_a = 240$ V}
	\label{fig:240}
\end{figure}

Na wykresie napięcia twornika widać, że napięcie to rośnie w czasie. Jest to konsekwencją występowania siły elektromotorycznej (SEM), która indukuje napięcie przeciwdziałające zmianom prądu. W rezultacie średnia wartość napięcia pozostaje nieco wyższa od wartości SEM, co zapewnia przepływ prądu w uzwojeniach wirnika. Obserwowany ciągły wzrost napięcia odpowiada narastaniu SEM wraz z przyspieszaniem wirnika w trakcie symulacji.




\vspace{1.5cm}
\section{Porównanie stanu dynamicznego rozruchu}
% zadanie 4


Porównanie przebiegów stanu dynamicznego rozruchu dla idealnego modelu w środowisku Simulink oraz dla rzeczywistego silnika sterowanego przekształtnikiem DML ujawnia istotne różnice w zakresie dynamiki i wartości szczytowych prądu.

\subsection{Zakres rozbieżności}

\begin{itemize}
	\item \textbf{Dynamika prądu twornika ($\mathbf{i_a}$):}
	
	W modelu Simulink (np. rysunek \ref{fig:sym50}), reakcja na skok napięcia jest gwałtowna. Prąd rozruchowy rośnie bardzo szybko, a jego wartość teoretyczna jest znacząco przeszacowana.
	Dla przykładu, w symulacji dla $U_a = 200\,\text{V}$ prąd może osiągać nawet $50\,\text{A}$. W rzeczywistym silniku (np. rysunek \ref{fig:50}d), prąd ma charakter impulsowy, osiągając szczyty rzędu $12$--$14\,\text{A}$. Średnia wartość prądu w fazie rozruchu jest dużo niższa, sięgając jedynie około $3.5\,\text{A}$. W obu przypadkach po fazie rozruchu prąd ustala się na niskich wartościach.
	
	\item \textbf{Dynamika prędkości obrotowej ($\omega_r$):}
	
	W modelu Simulink prędkość obrotowa narasta gwałtownie (bez przeregulowań) i bardzo szybko stabilizuje się na wartości ustalonej (np. rysunek \ref{fig:sym50}).
	W rzeczywistym silniku (np. rysunek \ref{fig:50}b, \ref{fig:50}d) narastanie prędkości jest wyraźnie wolniejsze i bardziej liniowe. Charakterystyka ta odzwierciedla mniejszą dynamikę sterowania przez przekształtnik oraz rzeczywiste parametry mechaniczne układu.
\end{itemize}

\subsection{Przyczyny rozbieżności}

Zaobserwowane różnice pomiędzy symulacją a pomiarami wynikają przede wszystkim z tego, że model obwodowy w Simulinku jest znacznym uproszczeniem rzeczywistego układu. Model traktuje silnik jak idealny układ RL, który reaguje natychmiast i bez żadnych strat, podczas gdy prawdziwy napęd zachowuje się o wiele bardziej złożenie. W praktyce silnik zasilany jest poprzez przekształtnik DML, który mimo dużej dynamiki nie jest w stanie odwzorować przebiegów w sposób tak idealny jak symulacja. Przekształtnik generuje dodatkowo pulsacje prądu, dobrze widoczne na przebiegach pomiarowych, czego model nie uwzględnia.

Do tego dochodzi fakt, że model symulacyjny pomija wiele zjawisk fizycznych występujących w realnym silniku, takich jak zużycie i nierównomierność pracy szczotek, niewielkie przesunięcia komutatora względem wirnika czy rezystancje przewodów i straty mocy w samym przekształtniku. W warunkach laboratoryjnych znaczący wpływ mają także zakłócenia elektromagnetyczne, które powodują zaszumienie sygnałów i utrudniają bezpośrednie porównanie przebiegów.

Podsumowując, rozbieżności te są naturalne i potwierdzają, że model w Simulinku jest jedynie przybliżeniem rzeczywistego układu. Rzeczywisty napęd charakteryzuje się mniejszą dynamiką, a jego zachowanie zależy również od zjawisk związanych z komutacją i sterowaniem impulsowym, których prosty model obwodowy nie jest w stanie w pełni odwzorować.


\vspace{1.5cm}
\section{Przykładowa aplikacja silnika}
% zadanie 5

Wybór silnika zależy od koncepcji całego układu napędowego. Proces ten można jednak sprowadzić do kilku podstawowych kryteriów:
\begin{enumerate}[label*=\alph*)]
	\item napiecie i rodzaj prądu: DC lub AC
	\item moc i prędkość obrotowa (kątowa)
	\item rodzaj pracy znamionowej 
\end{enumerate}
Zatem, rozważając przykładowe zastosowanie silnika analizowanego podczas ćwiczenia, zwrócono szczególną uwagę na powyższe kryteria \cite{zawirski}.
\\

Silnik PZBb44b to maszynowy silnik prądu stałego o mocy znamionowej 1.5 kW,
zasilaniu 230 V oraz prądzie twornika 6.5 A. Pracuje przy prędkości
1450~rpm i posiada wzbudzenie własne o prądzie 0.4~A. Parametry te pozwalają
na wykorzystanie go w aplikacjach wymagających stabilnej prędkości obrotowej
i umiarkowanego momentu.

\subsection{Aplikacje}

Ze względu na moc i prędkość znamionową analizowany silnik może być stosowany w:
\begin{itemize}
	\item małych pompach wirowych,
	\item sprężarkach tłokowych o niskiej mocy,
	\item przenośnikach taśmowych,
	\item mieszadłach i mieszarkach przemysłowych,
	\item prostych napędach maszyn warsztatowych.
\end{itemize}

\subsection{Rozpoznanie rynku}

\begin{table}[h!]
	\centering

	\begin{tabular}{p{3cm} p{1.2cm} p{2cm} p{3.5cm} p{3cm} p{2cm}}
		\toprule % Najgrubsza linia na górze
		\textbf{Model} & \textbf{Moc} & \textbf{Prędkość} & \textbf{Napięcie} & \textbf{Prądy} & \textbf{Moment} \\ 
		\midrule % Średnia linia oddzielająca nagłówek
		
		PZBb44b & 1.5 kW & 1450 rpm & 230 V (DC) & 6.5 A (tw.), 0.4 A (wzb.) & $\approx 9.9$ Nm \\ 
		\rowcolor[gray]{0.95} % Szary kolor wiersza
		ABB M3AA & 1.5 kW & 1445--1450 rpm & 230/400/415 V, 3F, 50 Hz & 3.24--5.85 A & $\approx 9.95$ Nm \\ 
		Siemens SIMOTICS GP & 1.5 kW & $\approx 1450$ rpm & 230/400 V lub 400/690 V, 3F & $\approx 8.4$ A (zależnie od wersji) & --- \\ 
		\rowcolor[gray]{0.95} % Szary kolor wiersza
		WEG W22 & 1.5 kW & 1450--1500 rpm & 220--240 / 380--415 V, 3F & --- & --- \\ 
		Maedler / SM-I IEC 90L & 1.5 kW & 1430--1450 rpm & 230/400 V, 3F & --- & --- \\ 
		\bottomrule 
		

	\end{tabular}
	\caption{Porównanie parametrów silników elektrycznych}
\end{table}

Silniki te stanowią porównywalne konstrukcje pod względem mocy i prędkości,
przez co mogą być wykorzystane jako modele referencyjne w procesie symulacji
i doboru parametrów zastępczych. Główne różnice pomiędzy silnikiem PZBb44b a pozostałymi silnikami
branymi pod uwagę wynikają przede wszystkim z odmiennych metod zasilania, a co za tym idzie
z charakterystycznych sposobów regulacji prędkości oraz kształtowania
momentu. Silnik PZBb44b jako maszyna prądu stałego z wzbudzeniem własnym
utrzymuje prędkość zależną od napięcia twornika oraz strumienia
wzbudzenia, co umożliwia płynną i szeroką regulację obrotów bez konieczności
stosowania dodatkowych układów energoelektronicznych. Z kolei silniki
indukcyjne AC pracują przy prędkości wyznaczonej przez częstotliwość sieci
i liczbę biegunów, a zmiana prędkości wymaga użycia falownika. W efekcie
każda z tych maszyn cechuje się odmiennym zachowaniem dynamicznym,
sprawnością regulacji oraz zakresem możliwej kontroli momentu, co przekłada
się na praktyczne zastosowania i sposób integracji z układem napędowym.


\vspace{1.5cm}
\section{Zbieżność modelu z danymi znamionowymi}
% zadanie 6
Model silnika nie jest idealny, a podczas doboru parametrów symulacyjnych konieczne jest określenie priorytetów, czy ważniejsze jest uzyskanie dokładnego odwzorowania prądu, czy prędkości obrotowej. W praktyce decyzja ta stanowi kompromis i zależy zarówno od planowanego zastosowania silnika, jak i preferencji projektanta.

W przedstawionym modelu udało się uzyskać zbieżność parametrów na poziomie akceptowalnym dla typowych analiz inżynierskich. Niewielkie odchylenia wynikają z uproszczeń modeli matematycznych oraz z pominięcia niektórych zjawisk fizycznych, takich jak nieliniowe straty w rdzeniu, starzenie materiałów czy zmienność parametrów uzwojeń przy nagrzewaniu. 

\begin{figure}[h!]
	\centering
	\includegraphics[width=0.8\textwidth]{230.pdf}
	\caption{Przebiegi czasowe prędkości obrotowej oraz prądu twornika uzyskanie symulacyjnie}
	\label{fig:zbieznosc}
\end{figure}

\begin{table}[h!]
	\centering

	\begin{tabular}{l c c}
		\toprule
		\textbf{Parametr} & \textbf{Znamionowe} & \textbf{Symulacja} \\ 
		\midrule
		Napięcie (V) & 230 & 230 \\ 
		Prędkość (rpm) & 1450 & 1428 \\ 
		Prąd (A) & 6.5 & 6.66 \\ 
		Moc (W) & 1500 & 1518 \\ 
		\bottomrule
	\end{tabular}
	\caption{Porównanie parametrów znamionowych i symulacyjnych}
\end{table}



\vspace{1.5cm}

\begin{thebibliography}{9}
	\bibitem{ekurs}
	Materiały do ćwiczenia laboratoryjnego dostępne na platformie eKursy, Politechnika Poznańska, \texttt{https://ekursy.put.poznan.pl/mod/folder/view.php?id=3022724}
	\bibitem{zawirski}
	K. Zawirski, J. Deskur, T. Kaczmarek, \textit{Automatyka napędu elektrycznego}, Wydawnictwo Politechniki Poznańskiej, 2012
	\bibitem{sidorowicz}
	J. Sidorowicz, \textit{Napęd elektryczny i jego sterowanie}, Oficyna Wydawicza Politechniki Warszawskiej, 1997
\end{thebibliography}




\end{document}