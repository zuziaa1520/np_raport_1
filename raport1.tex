\documentclass[a4paper,onecolumn,twoside,10pt]{article}%{mwrep}

\usepackage{times}
\usepackage[utf8x]{inputenc}
\usepackage[T1]{fontenc}
\usepackage[polish]{babel}
\usepackage{lmodern} %Type1-font for non-english texts and characters
\usepackage{setspace}
\usepackage{enumitem}

\usepackage[cmex10]{amsmath}

%% Packages for Graphics & Figures %%%%%%%%%%%%%%%%%%%%%%%%%%

%\usepackage{bmpsize}

\usepackage{graphicx} %%For loading graphic files
\usepackage[pdf]{pstricks}
\usepackage{pst-all}
\usepackage{moredefs}
%\usepackage{auto-pst-pdf}
%\usepackage{auto-pst-pdf}
\usepackage[crop=off]{auto-pst-pdf}

\usepackage{fancyhdr}
\usepackage{url}
\usepackage{float}
\usepackage{color}
\usepackage{xcolor}

\usepackage{subfigure}
\usepackage{multirow}

%\usepackage{epstopdf}
\definecolor{light-gray}{gray}{0.4}
\definecolor{mauve}{rgb}{0.88, 0.69, 1.0}
\definecolor{pakistangreen}{rgb}{0.0, 0.4, 0.0}
\definecolor{pearl}{rgb}{0.94, 0.92, 0.84}
\definecolor{whitesmoke}{rgb}{0.96, 0.96, 0.96}
\definecolor{gray-pp}{rgb}{0.13, 0.6, 0.82}

\usepackage{colortbl}
\usepackage{listings}
\lstset{
	basicstyle=\footnotesize\ttfamily,
	columns=fullflexible,
	frame=single,
	breaklines=true,
	numbers=left, stepnumber=2, numbersep=5pt,
	numberstyle=\tiny\color{gray},
	keywordstyle=\color{blue},
	commentstyle=\color{pakistangreen},
	stringstyle=\color{mauve},
	backgroundcolor = \color{whitesmoke},
	breakatwhitespace=true,
	showspaces=false,                % show spaces everywhere adding particular underscores; it overrides 'showstringspaces'
	showstringspaces=false,          % underline spaces within strings only
	showtabs=false,                  % show tabs within strings adding particular underscores
	postbreak=\mbox{\textcolor{red}{$\hookrightarrow$}\space}
}



\cfoot{-~\thepage \textcolor{light-gray}{~| Strona~-}}

\hyphenpenalty=10000		% nie dziel wyrazów zbyt często
\clubpenalty=10000			% kara za sierotki
\widowpenalty=10000			% nie pozostawiaj wdów
\brokenpenalty=10000		% nie dziel wyrazów między stronami
\exhyphenpenalty=999999		% nie dziel słów z myślnikiem
\righthyphenmin=3			% dziel minimum 3 litery

\tolerance=4500
\pretolerance=250
\hfuzz=1.5pt
\hbadness=1450

\sloppy						% umacnia pozycję prawego marginesu

\setlength{\textwidth}{\paperwidth}
\addtolength{\textwidth}{-5cm}
\setlength{\textheight}{\paperheight}
\addtolength{\textheight}{-5cm}
\setlength{\oddsidemargin}{0cm}
\setlength{\evensidemargin}{0cm}
\topmargin -1.25cm
\footskip 1.4cm

\linespread{1.3}

\begin{document}
	\raggedbottom 
	%\input{hypernation}
%\input{title-page}
\setlength\extrarowheight{2pt}
\begin{table}[ht]
	\centering
	%\resizebox{\textwidth}{!}{%
		\begin{tabular}{|p{5cm}|p{7cm}|p{2cm}|}
			\hline
			%\rowcolor{gray}
			%\multicolumn{3}{|c|}{\textcolor[rgb]{1,1,1}{Scalenie trzech kolumn}}\\
			
			\multicolumn{2}{|c|}{\cellcolor{gray-pp}\textcolor[rgb]{1,1,1}{Politechnika Poznańska}}  & \multicolumn{1}{c|}{\multirow{3}{*}{\resizebox{15mm}{!}{\includegraphics{PP_znak_konturowy_CMYK.pdf}}}}\\ 
			\multicolumn{2}{|c|}{\cellcolor{gray-pp}\textcolor[rgb]{1,1,1}{Wydział Automatyki, Robotyki i Elektrotechniki}} & \\ 
			\multicolumn{2}{|c|}{\cellcolor{gray-pp}\textcolor[rgb]{1,1,1}{Instytut Robotyki i Inteligencji Maszynowej}} & \\ 
			\hline 
			\multicolumn{1}{|c|}{Dz>AiR>Sem5} & \multicolumn{1}{c|}{Napędy przekształtnikowe (NP)} & \multicolumn{1}{c|}{2025/26 (s.zim.)} \\
			\hline
			\textbf{Skład osobowy:} \par Zuzanna Andrzejak 159522 \par Jan Andrzejewski  000000 \par Mateusz Banaszak 000000 \par Piotr Bednarek  000000 & 
			\textbf{Silnik elektryczny i jego model obwodowy. Siły: Lorentz'a, elektrodynamiczna, elektromotoryczna. Zbieżność modelu.} 
			& Data wyk.:\par 20.11.2025\\
			\hline
			Grupa 1  & Ćwiczenie 1 & Zajęcia 1 \\
			\hline
		\end{tabular}%
		%}
\end{table}	
\setlength\extrarowheight{0pt}
\vspace{1.5cm}
\tableofcontents
\newpage

\section{Wprowadzenie}

Ćwiczenie laboratoryjne miało na celu analizę modelu obwodowego silnika obcowzbudnego komutatorowego prądu stałego. Model zastosowany podczas ćwiczenia był modelem zidealizowanym i nie stanowił dokładnego odwzorowania rzeczywistego ukłądu napędowego. 

Ćwiczenie przeprowadzono w sali C3 w budynku A22b, wyposażonej w zestawy laboratoryjne napędowe, panele sterowania oraz przyrządy pomiarowe. W trakcie zajęć wykorzystano równień środowisko symulacyjne MATLAB wraz z Simulink. 
\vspace{1.5cm}

\section{Wstęp teoretyczny}

Model zrealizowany podczas ćwiczenia można w przybliżeniu opisać z wykorzystaniem podstawowych zjawisk fizycznych. Obejmują one przede wszystkim zjawiska elektromagnetyczne, elektromotoryczne oraz elektrodynamiczne, które określają zależności między przepływem prądu, strumieniem magnetycznym a momentem elektromagnetycznym oraz momentami wynikającymi z dynamiki ruchomych części silnika.

\subsection{Siła Lorentz'a}
Za generowanie siły w silniku prądu stałego odpowiada siła Lorentz'a, która określa oddziaływanie pola elektrycznego i magnetycznego na poruszający się ładunek. Jest określana równaniem:

\begin{equation}
	\vec{F}_{l} = q(\vec{E} + \vec{v} \times \vec{B})
	\label{eq:lorentz}
\end{equation}
gdzie:
$\vec{F}$ to wektor siły Lorentz'a (N),
$q$ to ładunek elektryczny cząstki (C),
$\vec{E}$ to wektor natężenia pola elektrycznego (V/m),
$\vec{v}$ to wektor prędkości cząstki (m/s),
$\vec{B}$ to pseudowektor indukcji magnetycznej (T).


\subsection{Siła elektrodynamiczna}

W rozważaniach dotyczących pracy silnika prądu stałego można pominąć wpływ wektora natężenia pola elektrycznego, gdyż jest ono praktycznie jednorodne. Przewodnik z prądem, umieszczony w polu magnetycznym, doświadcza siły, której kierunek i wartość zależą od ustawienia przewodnika względem pola. 
Maksymalna wartość siły występuje, gdy przewodnik jest ustawiony prostopadle do pola magnetycznego, co zapewnia konstrukcja \textbf{komutatora}. 

W tych warunkach siła Lorentza przyjmuje postać:
\begin{equation}
	F_{l} = qvB
	\label{eq:lorentz_simple}
\end{equation}

Dla przewodnika o określonej długości, w którym przepływają ładunki tworzące prąd, siła Lorentza przyjmuje postać \textbf{siły elektrodynamicznej}:
\begin{equation}
	\vec{F}_{ed} = I \, \vec{l} \times \vec{B}
	\label{eq:sila_ed}
\end{equation}
gdzie: $I$ to natężenie prądu, $\vec{l}$ to wektor długości przewodnika.


\subsection{Moment elektromagnetyczny}

Moment elektromagnetyczny w silniku prądu stałego jest generowany w wyniku działania sił elektrodynamicznych \eqref{eq:sila_ed} na uzwojenia twornika i wyraża się wzorem:
\begin{equation}
	T_e = \vec{r} \times \vec{F}_{ed} = I \left[ \vec{r} \times \left( \vec{l} \times \vec{B} \right) \right]
	\label{eq:moment_em_dlugie}
\end{equation}
gdzie: $I$ to natężenie prądu (A), $\vec{l}$ to długość uzwojenia (m), $\vec{B}$ to wektor indukcji magnetycznej (T), $\vec{r}$ to ramię siły (m).
\\

Dla uproszczonego modelu silnika prądu stałego można wyróżnić tzw. \textbf{stałą momentową $k_{\Phi}$}, która zależy od parametrów konstrukcyjnych silnika. Wtedy moment elektromagnetyczny wyraża się zależnością:
\begin{equation}
	T_e = i_a \, k_{\Phi}
	\label{eg:moment_Te}
\end{equation}
gdzie: $T_e$ to moment elektromagnetyczny (Nm), $i_a$ to prąd twornika (A), $k_{\Phi}$ to stała momentowa (Nm/A).


\subsection{Siła elektromotoryczna}

Siła elektromotoryczna (SEM) powstaje wskutek ruchu przewodnika w polu magnetycznym. Przewodniki wirnika przecinają linie pola, co indukuje napięcie i powoduje powstanie siły przeciwdziałającej ruchowi elektronów, równoważącej działanie sił Lorentza \eqref{eq:lorentz}. Dzięki temu moment elektromagnetyczny nie prowadzi do nieskończonego przyspieszenia wirnika.  

Wartość SEM zależy liniowo od prędkości obrotowej wirnika i parametrów konstrukcyjnych:
\begin{equation}
	\varepsilon = k_{\Phi} \, \omega_{r},
	\label{eq:sem}
\end{equation}
gdzie:
$\varepsilon$ to siła elektromotoryczna (V), $\omega_{r}$ to prędkość obrotowa wirnika (rad/s), $k_{\Phi}$ to stała napięciowa silnika (V $\cdot$ s/rad).
\vspace{1.5cm}
 
 
\section{Realizacja modelu w środowisku symulacyjnym}
% zadanie 1

\section{Analiza rozruchu modelu do prędkości biegu jałowego}
% zadanie 2

\section{Analiza porównawcza stanu biegu jałowego}
% zadanie 3

\section{Porównanie stanu dynamicznego rozruchu}
% zadanie 4

\section{Przykładowa aplikacja silnika}
% zadanie 5

Wybór silnika zależy od koncepcji całego układu napędowego. Proces ten można jednak sprowadzić do kilku podstawowych kryteriów:
\begin{enumerate}[label*=\alph*)]
	\item napiecie i rodzaj prądu: DC lub AC
	\item moc i prędkość obrotowa (kątowa)
	\item rodzaj pracy znamionowej 
\end{enumerate}
Zatem, rozważając przykładowe zastosowanie silnika analizowanego podczas ćwiczenia, zwrócono szczególną uwagę na powyższe kryteria.


\section{Zbieżność modelu z danymi znamionowymi}
% zadanie 6


\section{Literatura}
\begin{thebibliography}{9}
	\bibitem{ekurs}
	Materiały do ćwiczenia laboratoryjnego dostępne na platformie eKursy, Politechnika Poznańska, \texttt{https://ekursy.put.poznan.pl/mod/folder/view.php?id=3022724}
	\bibitem{zawirski}
	K. Zawirski, J. Deskur, T. Kaczmarek, \textit{Automatyka napędu elektrycznego}, Wydawnictwo Politechniki Poznańskiej, 2012
\end{thebibliography}




\end{document}